\documentclass[12pt]{article}
\usepackage{graphicx}
\usepackage[latin1]{inputenc}   % para os acentos
\usepackage[brazil]{babel}      % para hifeniza\c{c}\~{a}o


\setlength{\oddsidemargin}{-0.5cm}
\setlength{\evensidemargin}{-0.3cm}\setlength{\textwidth}{17.6cm}
\setlength{\textheight}{24cm}\setlength{\topmargin}{-1.0cm}
\setlength{\headheight}{0.0cm} \setlength{\headsep}{0.0cm}

%\twocolumn

\pagestyle{empty}
\usepackage{listings}
\begin{document}

\lstset{basicstyle=\scriptsize,breaklines=true,frame=lines}
\lstloadlanguages{Java}
\lstset{language=Java}
\lstset{numbers=left, numberstyle=\scriptsize, stepnumber=1, numbersep=5pt}

\noindent
\rule{1\textwidth}{0.2mm}
\includegraphics[height=6mm, width=11mm]{logo_cin.jpg} \hfill \textbf{{\large Trabalho 07s}} \hfill \includegraphics[height=8mm, width=7mm]{logo_ufpe.jpg} \\
\rule{1\textwidth}{0.2mm}\\

\noindent
{\bf Disciplina:} Programa��o Paralela
\\[0.2cm]
\textbf{Data: 28 de abril de 2015.}
\\[0.2cm]
{\bf Prof.:} Fernando Castor\\[0.5cm]
1. Implemente um programa de contagem estat�stica (conforme descrito no Perfbook). Seu programa deve executar um n�mero {\em N} de {\bf threads contadoras} (TCs) parametriz�vel. Estas funcionam em um la�o, incrementando seus contadores locais a cada itera��o. Al�m disso, deve executar uma {\bf thread leitora} respons�vel por ler os valores atuais dos contadores das TCs e produzir uma soma global desses valores, imprimindo-a. A execu��o deve parar quando o valor da soma total atingir um limite {\em K}, tamb�m parametriz�vel. Execute esse programa para valores de $K$ maiores que $K\geq2^{31}$ e me�a o tempo de execu��o. Qual o efeito de tornar {\tt volatile} o contador de cada thread? O tempo de execu��o muda? O tipo do atributo contador influencia esse tempo de execu��o? Compare os resultados para {\tt int}, {\tt float}, {\tt double} e {\tt long}. Os intervalos entre leituras dos contadores de cada thread influencia o tempo total da execu��o? Para medir o tempo, realize pelo menos dez execu��es e use a m�dia das �ltimas tr�s execu��es como seu tempo oficial. 
\\[0.5cm]
\noindent
2. Verifique, para o programa do exemplo anterior, o efeito de usar contadores dos tipos {\tt AtomicInteger} e/ou {\tt AtomicLong} no tempo de execu��o.
\\[0.5cm]
\noindent
3. Torne exato (ou seja, n�o mais estat�stico) o contador do item 1. Torn�-lo exato significa que a soma total n�o pode passar de $K$. Uma abordagem de {\em fastpath} pode ajudar neste caso? Em caso afirmativo, mostre como. Em caso negativo, mostre porque n�o. Qual o desempenho desse contador? Empregue a mesma metodologia de medi��o descrita nos itens anteriores.
\end{document} 
